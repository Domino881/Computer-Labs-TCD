\documentclass[11pt]{article}
\usepackage[utf8]{inputenc}
\usepackage[margin=2.3cm, showframe]{geometry}
\usepackage{indentfirst}
\usepackage{tikz}
\usepackage{multicol}
\usepackage{graphicx}
\usepackage{wrapfig}
\usepackage{blindtext}
\usepackage{amsmath}
\usepackage{adjustbox}
\usepackage[version=4]{mhchem}
\usepackage{graphicx}
\usepackage{caption}
\usepackage{float}
\usepackage[section]{placeins}
\usepackage{adjustbox}
\usepackage{setspace}
\usepackage{listings}
\definecolor{capt}{RGB}{12,12,12}

\setstretch{1.2}

\captionsetup{
  format = hang,
  font = footnotesize,
  labelfont = {bf},
}

\setlength{\parskip}{5pt}
\setlength\parindent{0pt}

\addtolength{\jot}{1em}

\lstdefinestyle{myStyle}{
    belowcaptionskip=1\baselineskip,
    breaklines=true,
    frame=none,
    backgroundcolor=\color{gray!7},
}

% \lstset{framexleftmargin=5mm, frame=shadowbox, rulesepcolor=\color{blue}}


\title{\vspace{-2cm}\textbf{Experiment: Numerical solution of the time-independent 1-D Schr{\"o}dinger equation}}
\author{Dominik Kuczynski (student id: 21367544)}
\date{Sep-Oct 2023, Monday AM}

\begin{document}

\maketitle

\section{Abstract}

% include all results!

\section{Introduction}

\section{Theoretical Background}

The time-independent form of the Schr{\"o}dinger equation is:

\begin{equation}
  E \psi(x) = \frac{-\hbar^2}{2m}\frac{d^2\psi}{dx^2}+V(x)\psi(x)
  \label{eq:SE}
\end{equation}

Now, in considering a system contained within a finite region of space, say from
$x=0$ to $x=L$, and with a given potential and particle mass, it is useful to 
transform Equation \ref{eq:SE} into a dimensionless form:
\begin{align}
     &\frac{d^2 \psi(\tilde{x})}{d\tilde{x}^2}+\gamma^2 (\epsilon-\nu(\tilde{x})))\psi(\tilde{x})=0
     \label{eq:2}\\ 
  \iff &\psi''(\tilde{x})+k^2(\tilde{x})\psi(\tilde{x})=0\notag
\end{align}

where$\qquad\tilde{x}=x/L,\quad \epsilon=E/V_0,\quad \nu(\tilde{x})=V(\tilde{x})/V_0,\quad \gamma^2=\frac{2mL^2V_0}{\hbar^2}$.

To solve Equation \ref{eq:2} for the infinite potential well, write:
\begin{align*}
  V(\tilde{x})=
  \begin{cases}
    -V_0,\quad 0<\tilde{x}<1\\
    \ \infty,\quad \text{else}\\
  \end{cases}
  \implies \nu(\tilde{x})=
  \begin{cases}
    -1,\quad 0<\tilde{x}<1\\
    \infty,\quad \text{else}
  \end{cases}
\end{align*}

In the allowed ($0<\tilde{x}<1$) region, Equation \ref{eq:2} becomes:
\begin{equation}
  \psi''(\tilde{x})=-\omega^2 \psi
\end{equation}
where $\omega^2 = \gamma^2(\epsilon+1)$. The solution is of the form:
\begin{equation}
  \psi(\tilde{x})=c_1 e^{i \gamma \sqrt{\epsilon+1} x} +c_2 e^{-i \gamma \sqrt{\epsilon+1} x}
\end{equation}

The disallowed region imposes additional conditions that $\psi(\tilde{x})=0$
for $\tilde{x}\leq0$ or $\tilde{x}\geq1$. These as well as normalisation
allow for the solving of $c_1$ and
$c_2$, as well as restrict the values of $\omega$. The final solution becomes:
\begin{equation}
  \psi(\tilde{x})=\sqrt{2}\sin(\gamma \sqrt{\epsilon+1} \tilde{x}),\quad \gamma\sqrt{\epsilon+1}=2\pi n,\ n=1,2,...
\end{equation}
\begin{equation}
  \implies \epsilon=\frac{4\pi^2n^2}{\gamma^2}-1,\quad n=1,2,...
  \label{eq:epsilon}
\end{equation}

To solve for the harmonic potential, first consider the dimensionful solution when
$V(x)=\frac{1}{2}k x^2$\cite{grif}:
\begin{equation}
  E_n=(n+\frac{1}{2})\hbar \omega,
  \quad \psi_n(x)=\left(\frac{m\omega}{\pi \hbar}\right)^{1/4}\frac{1}{\sqrt{2^n n!}}H_n(\xi)e^{-\xi^2/2},
  \quad n=0,1,2,...
\end{equation}
where $\xi=\sqrt{\frac{m\omega}{\hbar}}x$. 

To convert this into the dimensionless form similar to one being analysed,
write $V(x)=\frac{1}{2}k(x-L/2)^2-V_0=\frac{1}{2}kL^2(\tilde{x}-0.5)^2-V_0$.
Since $\nu(\tilde{x})=V(\tilde{x})/V_0=8(\tilde{x}-0.5)^2-1$: 
\begin{equation}
  k=  \frac{16V_0}{L^2}
\end{equation}

This allows us to solve for $\epsilon_n$:
\begin{equation}
  \epsilon_n=E_n/V_0=(n+\frac{1}{2})\frac{\hbar}{V_0}\sqrt{\frac{k}{m}}-1=
  (n+\frac{1}{2})\frac{\hbar}{V_0}\sqrt{\frac{16V_0}{L^2m}}-1=
  (n+\frac{1}{2})\frac{4\sqrt{2}}{\gamma}-1
\end{equation}

Note that here $E$ was shifted down by $V_0$, together with $V(x)$.

\subsection{Uncertainty relation}

To calculate $\Delta \tilde{x}=\sqrt{\langle \tilde{x}^2\rangle-\langle\tilde{x}\rangle^2}$,
it is useful to note that for symmetric potentials, the particle is as likely to be 
"on the left" as it is "on the right" of the axis of symmetry, meaning $\langle \tilde{x}\rangle$
is precisely the position of that axis. In the cases examined here, $\langle x \rangle=0.5$.

The second moment of $\tilde{x}$ can be calculated using
\begin{equation}
  \langle\tilde{x}^2\rangle=\int_0^1 \tilde{x}^2|\psi(\tilde{x})|^2d\tilde{x}
\end{equation}

To find the uncertainty $\Delta \tilde{p}$, one needs to use that the 
non-dimensional momentum operator is $\hat{\tilde{p}}=-i\frac{d}{dx}$.
Then, $\langle \tilde{p}\rangle=\langle\psi^*|\hat{\tilde{p}}|\psi\rangle$=
$-i\int_0^1\psi\frac{d\psi}{d\tilde{x}}d\tilde{x}$ for real $\psi$. But since
$\psi(0)=\psi(1)=0$, integration by parts of the above yields that
$\langle\hat{\tilde{p}}\rangle=0$.

Similarly, $\langle \tilde{p}^2\rangle=\langle\psi^*|\hat{\tilde{p}}^2|\psi\rangle$=
$-\int_0^1\psi\frac{d^2\psi}{d\tilde{x}^2}d\tilde{x}$ and therefore:
\begin{equation}
  \Delta\tilde{p}=\sqrt{-\int_0^1\psi\frac{d^2\psi}{d\tilde{x}^2}d\tilde{x}}.
\end{equation}

\section{Method}

To solve the Schr{\"o}dinger equation numerically, the variable $\tilde{x}$ gets
discretized into $N$ values\\
 $x_n\ ,\ n=1,2,...,N$, between 0 and 1. Then, 
the value of $\psi$ at a given point $x_{n+1}$ can be approximated using the
values of $\psi$ and $\nu$ at two previous points:

\begin{align} 
  &\psi_{n+1}\approx\frac{2(1-\frac{5}{12}l^2k_n^2)\psi_n-(1+\frac{1}{12}l^2k^2_{n-1})\psi_{n-1}}{1+\frac{1}{12}l^2k^2_{n+1}}
  \\\label{eq:4}
  \text{with }&k_n^2=\gamma^2(\epsilon-\nu(\tilde{x}_n))
\end{align}

This allows for the numerical integration of $\psi(\tilde{x})$ from 0 to 1,
 given its values at the two
initial points $x_1$ and $x_2$.

To obtain $\psi$'s second derivative at a given point, the finite difference
scheme was used. Denoting by $l$ the separation of discrete points $x_n$:
\begin{equation}
  \psi_n''\approx\frac{\psi_{n-1}-2\psi_n+\psi_{n+1}}{l^2}
\end{equation}

When the value $\epsilon$ is given, a numerical integration algorithm, following
(\ref{eq:4}) can 
generate a wavefunction, provided two initial conditions.
For the purposes of this lab, the conditions were the following:
\begin{equation*}
  \psi(x_1)=0,\quad \psi(x_2)=10^{-4}
\end{equation*}
$x_0$ and $x_1$ correspond to the first and second sampling points of the
numerical algorithm and the value $\psi(x_1)$ is chosen to be arbitrarily non
zero.

\subsection{Shooting method for finding energy eigenstates}

Since the energy spectrum of the infinite well potential is discrete, only specific
values of energy will produce normalizable (physical) wavefunctions. 
The method used for finding these values makes use of the fact that non-eigenstate
energies will produce wavefunctions blowing up to large positive or large negative
values at the end of the interval, due to the exponential nature of the Schr{\"o}dinger
equation. However, since the final result is continuous in $\epsilon$, 
between two energies producing $\psi$ blowing
up with opposite signs, there must be one for which $\psi(1)=0$.

Therefore, the final method changes the value of examined $\epsilon$
by smaller and smaller corrections to find the energy at which $\psi(1)$ goes
from negative to positive or vice versa, not unlike the bisection method for
finding roots of functions. The stopping condition is that the corrections
become smaller than a desired tolerance.

\begin{lstlisting}[language=Python, style=myStyle]
  def shoot(energy, v, tol=5e-4, de=5e-3):
      # initialize e to a starting value
      e = energy

      # variable for storing psi(1) of the previous iteration,
      # used for determining whether the sign has changed
      prevLastPsi = 0
      
      while(abs(de) > tol):
          psi = integratePsi(e, v)
          psi = normalize(psi)
          
          # if the sign of psi(1) has changed, move to in between 
          # the last two energies
          if psi[-1] * prevLastPsi < 0:
              de = -de/2

          prevLastPsi = psi[-1]
          e += de

      return e
\end{lstlisting}

\section{Results \& Discussion}

\subsection{Infinite well potential}

The wavefunctions of the first three found eigenstates were plotted to
obtain the following:

\begin{figure}[htbp]
  \centering
  \includegraphics*[width=0.6\textwidth]{ex_eigenstates1.png}
  \caption{First three generated eigenstates of the infinite well potential}
\end{figure}


\begin{wrapfigure}[4]{R}{0.6\textwidth}
  \vspace{-1cm}
  \centering
  \includegraphics*[width=0.6\textwidth]{error_sin.png}
  \caption{Absolute error between the generated and analytic solution}
\end{wrapfigure}

\vspace{1cm}

This result appears to correspond to the analytic solutions. To actually
check the accuracy, the error in the first eigenstate was plotted and found to
be less than 0.005.

\begin{table}[h]
  \centering
  \caption{Normalised eigenstate energies $\epsilon_n$ - infinite well}
  \label{tab:1}
  \begin{tabular}{c|c|c|c}
    n & numerical result (\texttt{tol=1e-5}) & analytic result & absolute error\\
    \hline
    1 & -0.95055 & -0.95065 & $0.00010$\\
    2 & -0.80220 & -0.80260 & $0.00040$\\
    3 & -0.55498 & -0.55586 & $0.00088$\\
    4 & -0.20884 & -0.21043 & $0.00159$\\
    5 &  0.23616 &  0.23370 & $0.00247$\\
    6 &  0.78008 &  0.77652 & $0.00355$\\
    7 &  1.42290 &  1.41805 & $0.00485$\\
    8 &  2.16459 &  2.15827 & $0.00632$\\
    9 &  3.00518 &  2.99718 & $0.00800$\\
    10 &  3.94469 &  3.93480 & $0.00989$

  \end{tabular}
\end{table}

The obtained eigenstate energies were compared with the analytical
results for a given tolerance and the errors were calculated
(Table \ref{tab:1}).

To further investigate how the tolerance 
value affects the accuracy, the errors of the first three energies were
plotted for different tolerances:

\begin{figure}[h]
  \centering
  \captionsetup{width=0.6\textwidth}
  \includegraphics*[width=0.6\textwidth]{error_vs_tol.png}
  \caption{Absolute error between numerical and analytic values of the first
  three eigenstate energies - for different values of tolerance}
\end{figure}

It is clear that the tolerance value affects the resulting accuracy significantly.
However, the above figure also shows the existence of other sources of error - 
which have the dominant effect when tolerance is low enough.

It seems most likely that the main source of these errors would be the accuracy of
the numerical integration. This is consistent with the observed relation where 
higher energies, corresponding to more rapidly oscillating and thus more error-prone
wavefunctions, have larger errors.

The uncertainty relation for the infinite well potential was checked by plotting 
$\Delta x \Delta p$ at the calculated eigenstate energies $\epsilon_n$:

\begin{figure}[h]
  \centering
  \includegraphics*[width=0.6\textwidth]{uncertainty1.png}
  \caption{Uncertainty relation - infinite well}
\end{figure}

\subsection{Bounded harmonic potential}

For $\nu(x)=8(x-0.5)^2-1$, the first three calculated eigenstates were the following:

\begin{figure}[h]
  \centering
  \includegraphics[width=0.6\textwidth]{ex_eigenstates2.png}
  \caption{First three generated eigenstates of the bounded harmonic potential}
\end{figure}

The calculated eigenstate energies were, again, compared to the analytical solutions.
However, since there is no analytic solution for the harmonic potential bounded by
infinite walls, the solutions of the unbounded harmonic potential was used to compare.
The discussion of this approximation can be found in the later part of this section.

\begin{table}[htbp]
  \centering
  \caption{Normalised eigenstate energies $\epsilon_n$ - bounded harmonic potential}
  \begin{tabular}{c|c|c|c}
    n & numerical result (\texttt{tol=1e-9}) & analytic result & absolute error\\
    \hline
    1 &-0.910557280&-0.910557281 & 0.000000001\\
    2 &-0.731671835&-0.731671843 & 0.000000008\\
    3 &-0.552786242&-0.552786405 & 0.000000163\\
    4 &-0.373898901&-0.373900966 & 0.000002065\\
    5 &-0.194996989&-0.195015528 & 0.000018539\\
    6 &-0.016005948&-0.016130090 & 0.000124141\\
    7 & 0.163395485& 0.162755348 & 0.000640137\\
    8 & 0.344230419& 0.341640786 & 0.002589633\\
    9 & 0.528873026& 0.520526225 & 0.008346801\\
    10& 0.721278843& 0.699411663 & 0.021867180
  \end{tabular}
  \label{tab:2}
\end{table}

Again, like in the case of the infinite well, it is clear that there are other
sources of error than the set tolerance - particularly at higher energies.

The uncertainty relation was plotted next:

\begin{figure}[h]
  \centering
  \includegraphics[width=0.6\textwidth]{uncertainty2.png}
  \caption{Uncertainty relation - bounded harmonic potential}
\end{figure}

This time, 20 of the first eigenstate energies were used.

To see the discrepancy with the analytic solution of the unbounded harmonic potential,
it is particularly useful to look at higher energy eigenstates. That is because
at high enough energies, the system effectively starts behaving like an infinite
square well.

One way to observe this is to plot a few examples of high-energy generated eigenstates:

\begin{figure}[h]
  \centering
  \includegraphics[width=0.6\textwidth]{ex_eigenstates3.png}
  \caption{Example high-energy eigenstates of the bounded harmonic potential}
\end{figure}

It is clear that their behaviour is more like that of the infinite well eigenstates,
as they appear to have roots at $x=0$ and $x=1$, instead of identically 
converging to 0.

Another way of clearly observing the discrepancy, is to look at differences between
consecutive eigenstate energies calculated by the algorithm: 

\begin{figure}[h]
  \centering
  \includegraphics[width=0.6\textwidth]{diff_energies.png}
  \caption{Differences between consecutive energies (log-log) - bounded harmonic potential}
\end{figure}

For low values of $n$, the energies are evenly spaced - in accordance with the 
harmonic potential. However, for $n$ higher than around 10 the plot shifts into the
infinite well region - and produces a line with the slope of 2, as predicted by
that type of potential.

\section{Conclusion}

\section{Appendix}

\listoffigures

\subsection*{Bibliography}

\bibliography{refs}
\bibliographystyle{plain}

\end{document}